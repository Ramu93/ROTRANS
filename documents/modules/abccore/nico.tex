\subsection{Local Data Handling}
\label{local_data_handling}
In this section, Nico explains the parts of the group project he were mainly responsible for, and how they affected the intermediate and final versions of the project.

In the project group ROTRANS, Nicos part was to work mainly with the DAG representation, its corresponding data and handling the local data for the main actor of the project, the Agent. To get a better understanding of what Nicos part in the group was, we start this chapter by visiting the core of our project representation, the DAG. After this, we will provide an abstract overview over the functions maintained by Nico and then go into detail on the checkpoint injection process. The last part of this chapter will be a personal note by Nico on the project and this module.

\paragraph{DAG.py} contains the general structure of the data objects handled in our project and provides small functions related to these objects. In a granular view, the Wallet object is one of the most important parts of the project, as it holds the actual money of the system. For explanation purpose, a Wallet can be seen as a piggy bank with its owners name on it, meaning that it can ever only be opened once to take money out. For functional purpose, the Wallet holds information about one of the public keys of its owner, its own value and state, and is uniquely identifiable.
Any Wallet will be contained in an instance of the class Node, which is the superclass of Transaction, Acknowledge, Genesis and Checkpoint.

A Transaction is the next larger object we need to handle in our project, when it comes to representation. Naturally, a Transaction holds one or multiple Wallets to spent and again one or multiple Wallets which are the result of the transaction. Besides the input and output Wallets, a Transaction also holds information about a delegated Validator, which gains the value of this transaction as stake in the next Checkpoint, and one cryptographical signature for each Wallet in the list of spent Wallets. We speak of a valid Transaction, if and only if for every Wallet in the input list, there is one valid signature signed with the secret key of that Wallets owner.

Closely related to the Transaction, an Acknowledge object represents the vote of a user confirming the correctness of one Transaction. For this, an Acknowledge holds as information the unique identifier of the corresponding Transaction and a signature. An Agent will always created one Acknowledge for each key pair he owns. An Acknowledge also holds the unique identifier of the previous Acknowledge signed with the same key, allowing to track back all Transactions acknowledged by one key.

To provide each Agent instance an initial wallet distribution our cryptocurrency uses an hardcoded genesis, loaded and saved in the Genesis object. It represents the initial wealth and stake distribution in our system and therefore holds Wallets for the initial Agents to use.

Lastly, to enable users to join the system at any time, we introduced Checkpoints to our system. This class was constructed and maintained by our dedicated subgroup. 

\paragraph{In an abstract view,} the file \texttt{agent.py} is the "brain" of our application. The file was mostly developed in close partnership of Matthias, Amin and Nico. One of the first things done in the Agent is recovering its memory, by using the \texttt{load\_data()} function. 

The \texttt{load\_data()} function starts a process to restore a previous session of the user by receiving and parsing data from a SQL database, implemented in \texttt{save\_handler.py}. The user specific data will be saved mostly in the data layer implemented in \texttt{AgentData.py.} The DAG is stored in a dedicated datastructure implemented in \texttt{prefix\_tree.py}

After data from a previous session has been restored and the Agent connected the cryptocurrency system, the user can participate by issuing transactions. The obvious action for this is by using the function \texttt{\_\_send\_money()}, implemented by Matthias. The function uses subroutines on the data layer for assuring the correctness and consistency of the Transactions, namely the functions \texttt{get\_transaction\_set(value)}, \texttt{outputs\_helper()} and \texttt{\_\_create\_signature()}. The Transaction object generated there is then published to the network.

By this point, the Agent is probably already participating as a Validator by receiving Transaction objects from the network and acknowledging valid ones. For this, the function \texttt{\_\_add\_transaction()} checks the consistency and correctness of any incoming Transaction and decides if the Transaction will be declined or added to a list of yet unconfirmed Transactions. Transactions, which are based on yet unconfirmed or unknown Transactions will be added to a list of orphaned Transactions instead.

Any Transaction that is added to the list of unconfirmed Transactions and which is considered valid, will then be acknowledge by the Agent, using the function \texttt{\_\_acknowledge()}. This function leverages the function \texttt{acknowledge()} of the \texttt{AgentData} to create or retrieve Acknowledge objects corresponding to the Transaction. These Acknowledges are then sent over the network.

Similar to receiving Transactions from the network, the Agent can receive Acknowledge objects. These are handled in the \texttt{\_\_add\_acknowledgement()}. After validation of the signatures in the Acknowledge object, the function checks if the corresponding Transaction is in the list of unconfirmed Transactions. In that case, the Acknowledge is counted for that Transactions needed stake. If a Transaction gains enough stake like this to be considered confirmed, the function \texttt{\_\_add\_confirmed\_trans()} removes the Transaction from the list of unconfirmed Transactions and adds it to the DAG representation in memory and the local database.

An Agent continually tries to guarantee consistent data by sending requests to the network. The function \texttt{\_\_retry\_missing\_txn\_request()} is used to regularly ask for Transactions missing to process unconfirmed Transactions and unprocessed Nodes.

Another major part of the Agent is to handle Checkpoints. The checkpoint service calls the function \texttt{inject\_checkpoint()} of the Agent for this. This function then starts the transition process by calling the function \texttt{switch\_to\_ckpt()}, which itself initates the transformation of the DAG to the new checkpoint state. While the data layer transforms the DAG, the latter function reevaluates Nodes which were already processed but not part of the Checkpoint and sends requests for unknown Transactions which were mentioned in the Checkpoint. After \texttt{switch\_to\_ckpt()} is done, the function \texttt{inject\_checkpoint()} recalculates the stake threshold new Transactions need to reach to be confirmed, based on the new checkpoint. 

Lastly, the Agent will regularly save the session data using the function \texttt{save\_data()}, leveraging the function of same name of the data layer. Similar to the \texttt{load\_data()}, the data will be parsed and then saved in a SQL database.

\paragraph{Checkpoint injection} is one of the most complex tasks of the Agent. Besides reevaluating Nodes which weren't included in the Checkpoint, there are a lot of other tasks to be done to ensure a consistent transition. To emphasize this, we will describe the routines needed for this transition in more detail, starting with a view behind the curtain of the data layer. Code~\hyperref[pseudo_transform_dag_1]{3.1} shows one half of an abstract pseudo code of the function \texttt{transform\_dag()} of the class \texttt{AgentData}.

\begin{center}
\label{pseudo_transform_dag_1} \small \samepage
\begin{verbatim}
TRANSFORM_DAG:   
  Delete current save of this session

  Create a new tree and add all previous checkpoints

  Add unspent TXNs to the tree
  For all unspent transaction outputs in the checkpoint:
     Add Wallet to utxo_set search for its dependend nodes

     Check if TXN of this Wallet is in tree or put it in set 
     missing_txn_requests
     
     Set this Wallet to UNSPENT

     Set the TXNs parents to this ckpt

  Add TXN.identifier to set of kept TXNs and add the TXN to 
  the new tree

  Add Checkpoint to new tree
...
\end{verbatim}	
\vspace{2mm}	
\footnotesize{Code~\hyperref[pseudo_transform_dag_1]{3.1}: First part of the pseudo code of function \texttt{transform\_dag()}.}
\end{center}

The first part here is to delete the currently used database. This is not the backup created by the function \texttt{switch\_to\_cpkt()}, but the database which is saved to and load from per default.
In this part of the functions pseudo code, the Transactions of the old tree will be copied to the new tree, only if they are marked as (partially) unspent by the Checkpoint. By keeping the original Transactions, we can ensure that no adversarial party in the checkpoint committee can manipulate the unspent Transactions, since all Transactions are signed once for each PK of one of its input Wallets. Also, for these Transactions, the attribut parents will be set to the Checkpoints unique identifier, establishing a line of trust from this checkpoint on, because the past of this Transaction will probably not be available after the transition.
The second part of this pseudo code describes that unspent Wallets will be used to search for their dependend nodes. This will be a part in the second half of the pseudo code. Also, All Wallets mentioned in the Checkpoint will be set to \texttt{UNSPENT}, ensuring that reevaluation of confirmed but not in the Checkpoint included Transactions will succeed.

\begin{center}
\label{pseudo_transform_dag_2} \small \samepage \centering
\begin{verbatim}
...
  For all unspent transaction outputs:
     If output belongs to one of this Agents keys, add it 
     to its balance

  For all rewards in the Checkpoint:
     If reward belongs to one of this Agents keys, add it 
     to its balance

  Search for TXNs which were confirmed while the ckpt was 
  created and mark them as deconfirmed

  For all deconfirmed Nodes in the old tree
     Add the TXNs to set pending_trans and reset them to UNSPENT

     Add the ACKs to set pending_acks

  For all TXNs in the old pending_transactions
     Add received ACKs to pending_acks

     Add TXNs to pending_trans

     Add all acks for this TXNs outputs and inputs to 
     mapping retained_wallets

  Override local acked_wallets with retained_wallets

  Override local tree with new_tree

  For each key of this Agent, set the last_ack to the 
  new Checkpoint

  Return pending_trans, pending_acks, missing_txn_requests
\end{verbatim}
\vspace{2mm}	
\footnotesize{Code~\hyperref[pseudo_transform_dag_2]{3.2}: Second part of the pseudo code of function \texttt{transform\_dag()}.}
\end{center}

Code~\hyperref[pseudo_transform_dag_2]{3.2} shows the second half of the pseudo code of the function  \\\texttt{transform\_dag()} of the class \texttt{AgentData}. In the first part of Code~\hyperref[pseudo_transform_dag_2]{3.2}, the balance of the Agent will be recomputed over the unspent transaction outputs and the rewards listed in the Checkpoint. This ensures, that the users balance is kept consistent with the Transactions confirmed directly after the transition. Note that this means, that if the user had spent Wallets contained in the Checkpoint prior to its injection, those Wallets will now be added back to the Agents balance. This is necessary to enable the correct reevaluation of the made Transactions and does not threaten the correctness of the system as detailed below. 
As already mentioned, the Transactions which were confirmed while the Checkpoint was created will be labeled deconfirmed and added to a set of \texttt{pending\_txns}, with their Wallets set to \texttt{UNSPENT}. Also, all Transactions which were pending before will also be added to that set for reevaluation. To ensure correctness and to consistency of the system, all Acknowledges of Transactions in the set \texttt{pending\_txns} are added to a set of \texttt{pending\_acks}. This will also be reevaluated, directly after this function finishes, to ensure that the user can not spent Wallets twice by accident. Also in this part of the pseudo code, it is mentioned that the mapping \texttt{retained\_wallets} will override the mapping \texttt{acked\_wallets}. The latter one maps a Wallet to a list of Acknowledges, and holds mappings for all Wallets of Transactions, which the Agent had acknowledged. This is a list, because for each Transaction, the Agent will create one Acknowledge for each of its key pairs. \texttt{retained\_wallets} holds only mappings of those Wallets which can still be accessed after the transition, meaning all inputs and outputs of Transactions in the new tree or in the \texttt{pending\_txns}.
Then, to ensure correctness and a line of trust in the future Acknowledges of this Agent, the attribute \texttt{last\_ack} will be set to this Checkpoints identifier for each key pair of this Agent.
Lastly, the sets \texttt{pending\_trans}, \texttt{pending\_acks} and \texttt{missing\_txn\_requests} will be returned.

\begin{figure}[htbp]
	\centering
	\label{fig:code_switch}
	\includegraphics[width=0.9\textwidth, trim={0cm 0cm 0cm 0cm},clip]{figures/nico/code_switch}
	
	\footnotesize{Figure~\hyperref[fig:code_switch]{3.3}: Python code of function \texttt{switch\_to\_cpkt()}.}
\end{figure}

Now, with the function \texttt{transform\_dag()} described above, Figure~\hyperref[fig:code_switch]{3.3} shows the actual Python code of the function \texttt{switch\_to\_cpkt()} of the Agent. 
This function first creates a backup of the session, before starting the transaformation on the data layer by calling the function \texttt{transform\_dag()} to receive the sets \texttt{pending\_trans}, \texttt{pending\_acks} and \texttt{missing\_txn\_requests}. After the transformation of the tree is done, the function calls a helper function to remove unprocessed Nodes, from the corresponding list named \texttt{orphaned\_nodes}, which were already in that list before the previous checkpoint was processed.
The next step is to clear the list of pending transactions, since all of those Transactions were added to the set \texttt{pending\_trans} in the function \texttt{transform\_dag()}. This ensures, that those Transactions will be evaluated in the next step, as if they were just received over the network. To quickly reconfirm Transactions that were deconfirmed in the transition process, the Acknowledges in the set \texttt{pending\_acks} will also be evaluated as if they were just received. To finish the process, this function requests the Transactions for all identifiers in the set \texttt{missing\_txn\_requests} and saves the session data in a new database.





























